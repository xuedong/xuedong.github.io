% Time-stamp: <squelette.tex   2 Feb 2007 17:44:19>

\documentclass[12pt]{article}

%\def\pdfshellescape{1}

% Pour les accents dans les fichiers
\usepackage[T1]{fontenc}        % Codage des fontes
\usepackage[latin1]{inputenc}   % Codage du fichier
\usepackage[francais]{babel}    % Pour du français
% Packages AMS très utiles
\usepackage{amssymb,amsfonts,amsthm,amsmath}
\usepackage{fullpage}
%\usepackage{auto-pst-pdf}

% Package pour la fabrication d'un index%\usepackage{makeidx}

% Pour les dessins d'automates éventuellement
% Il faut avoir télécharger gastex.sty et gastex.pro
% http://www.lsv.ens-cachan.fr/~gastin/gastex/gastex.html
\usepackage{gastex}
%\usepackage{vaucanson-g}

% Pour les algorithmes éventuellement
%\usepackage[plain]{algorithm}   
%\usepackage[noend]{algorithmic} 

% Package de dessin
\usepackage{pstricks,pst-node,pst-tree}

% Package pour avoir des liens hyper-texte dans les fichiers Pdf
% A mettre en dernier
% Pose parfois problème
%\usepackage[a4paper,ps2pdf]{hyperref}

% Définitions d'environements pour les théorèmes, lemmes, définitions, ...
% Tous les environnements partagent la même numérotation
\theoremstyle{plain}
\newtheorem{theoreme}{Th\'eor\`eme}
\newtheorem{lemme}[theoreme]{Lemme}
\newtheorem{proposition}[theoreme]{Proposition}
\newtheorem{corollaire}[theoreme]{Corollaire}
\newtheorem{conjecture}[theoreme]{Conjecture}
\theoremstyle{definition}
\newtheorem{definition}[theoreme]{D\'efinition}
\theoremstyle{remark}
\newtheorem{exemple}[theoreme]{Exemple}
\newtheorem{remarque}[theoreme]{Remarque}

\renewcommand{\theenumi}{\roman{enumi}{)}}
\renewcommand{\labelenumi}{\theenumi}

% Pour se servir des numérotations romanes
% \makeatletter
% \newcommand{\rmnum}[1]{\romannumeral #1}
% \newcommand{\Rmnum}[1]{\expandafter\@slowromancap\romannumeral #1@}
% \makeatother

% Les éléments suivants founiussent les éléments à la commande \maketitle
% Titre
\title{Suites automatiques}
% Auteur
\author{Xuedong Shang\\\'Ecole Normale Sup\'erieure}
% Date optionnelle
\date\today

\begin{document}

% Mise en place du titre
\maketitle

\tableofcontents

\begin{abstract}
Nous donnons ici, dans ce petit rapport, un premier traitement, tr\`es bref, des suites engendr\'ees par un automate, mod\`ele le plus simple de machine. Une suite automatique n'est pas n\'ecessairement tr\`es simple ou r\'eguli\`ere (ultimement p\'eriodique par exemple) malgr\'e le fait que cela soit d\^ue \`a un automate dont les \'etats sont finis. La prem\`ere partie de ce rapport est consacr\'ee \`a une premi\`ere aper\c cu des suites automatiques avec quelques exemples connus. Puis nous traiterons quelques caract\'erisations de l'automaticit\'e en cherchant les liens entre les automates et leurs suite engendr\'ee associ\'ee. \`A la fin, nous donnerons certains r\'esultats divers sur les suites automatiques.
\end{abstract}

\section{D\'efinitions et exemples}
On va commencer par d\'efinir la \textit{k}-automaticit\'e et donner deux exemples de suites 2-automatiques. Ces suites pr\'esenteront des similarit\'e dont les liens seront d\'evelopp\'es dans la partie suivante.

Les automates que l'on va utiliser sont d\'eterministes et complets. L'ensemble des \'etats sera not\'e $Q$ et au lieu d'avoir certains \'etats finaux on dispose d'une fonction de sortie $\tau$ qui \`a chaque \'etat associe un valeur "de sortie".

\begin{definition}
  Un automate fini est la donn\'ee de $\{Q,\Sigma,\delta,i,\tau,\Gamma\}$ o\`u $Q$ est l'ensemble des \'etats,
  $i\in{Q}$ est l'\'etat initial, $\Sigma$ est l'alphabet d'entr\'ee, $\delta$ la fonction de transition de
  $Q\times\Sigma$ dans $Q$. $\Gamma$ l'alphabet de sortie et $\tau$ la fonction de sortie de $Q$ dans  $\Gamma$.
\end{definition}

On remarque que pour obtenir un automate usuel, il suffit de prendre $\Gamma=\{0,1\}$ et de consid\'erer $q$ \'etat final si et seulement si $\tau(q)=1$. Dans la suite, on n'introduit plus l'alphabet de sortie $\Gamma$ dans l'automate s'il n'y pas d'ambigu\"it\'e.

\begin{definition} 
  Soit \textit{k} un entier $\ge$ 2. On dit qu'une suite ($u_{n})_n$ est \textit{k}-automatique s'il
  existe un automate $\cal{A}$ = $\{Q,\Sigma,\delta,i,\tau\}$ tel que :
  \begin{displaymath}
    \forall{n},\tau(\delta(i,\langle{n}\rangle_k)) = u_{n}
  \end{displaymath}
  o\`u $\langle{n}\rangle_k$ d\'esigne l'\'ecriture en base \textit{k} de n \textbf{\`a partir du
  chiffre de poids faible}. Si on lit dans l'automate $\cal{A}$ la suite constitu\'ee des 
  chiffres de $n$ en base $k$ \`a partir du chiffre de poids faible, on tombe sur un \'etat dont la 
  valeur de sortie est $u_{n}$. Il est tout \`a fait naturel de consid\'erer que la fonction de 
  sortie est \`a valeur dans le m\^eme ensemble que la suite et que les ar\^etes de l'automate sont 
  \'etique\'ees  par $\{0,1,\cdots,k-1\}$.
\end{definition}

\begin{remarque}
  Un tel automate est appel\'e $k$-automate, et on peut ainsi dire qu'une suite est $k$-automatique si elle est 
  engendr\'ee par un $k$-automate.
\end{remarque} 

\subsection{La suite de Thue-Morse}
La suite de Thue-Morse (appel\'ee souvent suite de Prouhet-Thue-Morse chez les francophones) fut d\'ecrite pour la premi\`ere fois par le math\'ematicien fran\c cais Eug\`ene Prouhet en 1851. Il la utilisa pour donner une solution \`a un probl\`eme de th\'eorie des nombres qui s'appelle le probl\`eme de Prouhet-Tarry-Escott. La suite fut red\'ecouverte par le math\'ematicien norv\'egien Axel Thue dans un article publi\'e en 1906 qui est l'article fondateur de la combinatoire des mots. Puis Marston Morse, en 1922, donna une nouvelle interpr\'etation de la suite, dans le cadre de la g\'eom\'etrie diff\'erentielle.

\begin{definition}
  On d\'efinit la suite de Thue-Morse, $(t_{n})_{n}$ de la fa\c con suivante :
  \begin{displaymath}
  \begin{array}{c}
    t_{0} = 0,\\
    \forall{n}\ge0, t_{2n}=t_{n},\\
    \forall{n}\ge0, t_{2n+1}=1-t_{n}
  \end{array}
  \end{displaymath}
  On donne ses premiers termes ci-dessous : 0110100110$\ldots$
\end{definition}

On voit que, d'apr\`es la d\'efinition de $t$, $t_{n}$ correspond \`a la somme des chiffres dans l'\'ecriture en base 2 de $n$ modulo 2. La suite de Thue-Morse est ainsi 2-automatique; elle est engendr\'ee par l'automate suivant\footnote{$q/n$ est la notation pour dire que la valeur de sortie $\tau(q)$ de l'\'etat $q$ est n.} :

\begin{figure}[!h]
 \begin{center}
   \unitlength=4pt
   \begin{picture}(25, 10)(0,-5)
   \gasset{Nw=5,Nh=5,Nmr=2.5,curvedepth=4}
   \thinlines
   \node[Nmarks=i,iangle=90](A1)(0,0){$q_{0}/0$}
   \drawloop[loopangle=180](A1){$0$}
   \node[Nmarks=if](A2)(25,0){$q_{1}/1$}
   \drawloop[loopangle=0](A2){$0$}
   \drawedge(A1,A2){$1$}
   \drawedge(A2,A1){$1$}
   \end{picture}
 \end{center}
 \caption{L'automate de Thue-Morse.}
\end{figure}

Donnons une autre mani\`ere pour d\'efinir la suite de Thue-Morse. On consid\`ere le morphisme 2-uniforme\footnote{Un morphisme est dit $k$-uniforme si l'image de toute lettre est un mot de taille $k$.} $\sigma : \{0,1\}^{*}\rightarrow\{0,1\}^{*}$ donn\'e par :
\begin{displaymath}
  \begin{array}{c}
    \sigma(0)=01 \\
    \sigma(1)=10
  \end{array}
\end{displaymath}
La suite $(\sigma^{n}(0))_{n}$ converge simplement vers $t$. Il en r\'esulte que $t$ est point fixe de $\sigma$.

\subsection{La suite de Rudin-Shapiro}
La suite de Rudin-Shapiro, aussi appel\'ee la suite de Golay-Rudin-Shapiro est une suite automatique nomm\'ee par Marcel Golay, Walter Rudin et Harold S.Shapiro, qui l'ont d\'ecourvert ind\'ependemment.

\begin{definition}
  On d\'efinit la suite de Rudin-Shapiro $(r_{n})_{n}$ de la fa\c con suivante :
  \begin{displaymath}
  \begin{array}{c}
    r_{0}=0, \\
    \forall{n}\ge0, r_{2n}=r_{n}, \\
    \forall{n}\ge0, r_{4n+1}=r_{n}, \\
    \forall{n}\ge0, r_{4n+3}=1-r_{2n+1}.
  \end{array}
  \end{displaymath}
  Ses premiers termes sont les suivants : 000100100001$\ldots$
\end{definition}

Il s'agit encore d'une suite 2-automatique. Elle est engendr\'ee par l'automate suivant :

\begin{figure}[!h]
 \begin{center}
   \unitlength=4pt
   \begin{picture}(75, 10)(0,-5)
   \gasset{Nw=5,Nh=5,Nmr=2.5,curvedepth=4}
   \thinlines
   \node[iangle=90](A0)(0,0){$q_{0}/0$}
   \drawloop[loopangle=180](A0){$0$}
   \node(A1)(25,0){$q_{1}/0$}
   \node(A2)(50,0){$q_{2}/1$}
   \node(A3)(75,0){$q_{3}/1$}
   \drawloop[loopangle=0](A3){$0$}
   \drawedge(A0,A1){$1$}
   \drawedge(A1,A0){$0$}
   \drawedge(A1,A2){$1$}
   \drawedge(A2,A1){$1$}
   \drawedge(A2,A3){$0$}
   \drawedge(A3,A2){$1$}
   \end{picture}
 \end{center}
 \caption{L'automate de Rudin-Shapiro.}
\end{figure}

Cette suite donne le nombre d'occurrences de "11" dans l'\'ecriture de $n$ en base 2, le tout modulo 2. On ne peut pas trouver comme pour la suite de Thue-Morse un morphisme 2-uniforme dont la suite soit point fixe. Il faut d'abord "agrandir" l'alphabet. D\'efinissons un morphisme $\sigma$ sur l'alphabet $\{A,B,C,D\}$ par :
\begin{displaymath}
  \begin{array}{c}
    \sigma(A)=AB \\
    \sigma(B)=AC \\
    \sigma(C)=DB \\
    \sigma(D)=DC
  \end{array}
\end{displaymath}
La limite simple de $\sigma^{n}(A)$ est de la forme suivante :
\begin{displaymath}
\begin{array}{c}
  ABACABDBABACDCAC\cdots
\end{array}
\end{displaymath}
La suite de Rudin-Shapiro s'obtient alors en prenant l'image du point fixe pour $\sigma$ par la projection $\pi:\{A,B,C,D\}\rightarrow\{0,1\}$ d\'efinie par :
\begin{displaymath}
\begin{array}{c}
  \pi(A)=\pi(B)=0 \\
  \pi(C)=\pi(D)=1
\end{array}
\end{displaymath}

\section{Th\'eor\`eme de Cobham}

\subsection{Le $q$-noyau}

Au niveau des ressemblances entre les deux suites, il est pertinent d'introduire certaines d\'efinitions.

\begin{definition}
  Soit $u$ une suite donn\'ee et $k\in\mathbb{N}$. le $q$-noyau que l'on notera d\'esormais 
  $K_{u}^{(q)}$ (ou plus simplement $K_{u}$ s'il n'y a pas d'ambigu\"it\'e) est :
  \begin{displaymath}
    K_{u}^{(q)}=\{(u(q^{\alpha}n+r))_{n}\ |\ \alpha\in\mathbb{N},\ 0\le{r}<\alpha\}
  \end{displaymath}
\end{definition}

Le $q$-noyau nous int\'eresse lorsqu'il n'est pas "trop gros" et qu'il contraint les suites \`a co\"incider avec leurs propres sous-suites extraites de pas $q^{k}$. Or il y a au plus un nombre d\'enombrable de telle suite, on s'int\'eresse alors au cas fini.

Lorsque le $q$-noyau est fini, il est facile de le calculer par machine. Il suffit de construire r\'ecursivement les ensembles $X_{k}$ donn\'es par : 
$$X_{0}=\{(a_{n})_{n\ge 0}\} et X_{k+1}=X_{k}\cup\{(b_{qn+r})_{n}|(b_{n})_{n}\in{X_{k}},r\in{[0,q-1]}\}.$$
S'il existe $k$ tel que $X_{k}=X_{k+1}$ alors c'est la valeur du $q$-noyau, ce dernier est alors fini.

\begin{proposition}
  Le 2-noyau de la suite de Thue-Morse est fini.
\end{proposition}

\begin{proof}
  On a $$X_{1}=\{(t_{2n})_{n},(t_{2n+1})_{n}\}=\{(t_{n})_{n},(1-t_{n})_{n}\}$$
  Donc $K_{t}=\{(t_{n})_{n},(1-t_{n})_{n}\}$ dont le 2-noyau est fini.
\end{proof}

\begin{proposition}
  Le 2-noyau de la suite de Rudin-Shapiro est fini.
\end{proposition}

\begin{proof}
  On a $$X_{1}=\{(r_{n})_{n},(r_{2n+1})_{n}\}$$
  Puis $$X_{2}=\{(r_{n})_{n},(r_{2n+1})_{n},(r_{4n+3})_{n}\}$$
  Ensuite $$X_{3}=\{(r_{n})_{n},(r_{2n+1})_{n},(r_{4n+3})_{n},(r_{8n+3})_{n}\}$$
  Enfin $$X_{4}=\{(r_{n})_{n},(r_{2n+1})_{n},(r_{4n+3})_{n},(r_{8n+3})_{n}\}=X_{3}$$
  D'o\`u le 2-noyau $K_{r}=\{(r_{n})_{n},(r_{2n+1})_{n},(r_{4n+3})_{n},(r_{8n+3})_{n}\}$ est fini.
\end{proof}

\subsection{Th\'eor\`eme de Cobham}

Donnons maintenant un lien avec les morphismes $q$-uniformes.

\begin{definition}
  Une suite $u$ \`a valeur dans un alphabet $\Sigma$ est dite \textit{l'image d'un point fixe 
  d'un point fixe d'un morphisme k-uniforme} $\sigma : \Sigma^{'}\rightarrow\Sigma^{'}$ s'il 
  existe $w\in\Sigma^{'\mathbb{N}}$ point fixe de $\sigma$ et $\pi$ projection de
  $\Sigma^{'}$ dans $\Sigma$ tels que $\forall{n}\in\mathbb{N},u_{n}=\pi(w_{n})$.
\end{definition}

En \'ecrivant le fait d'\^etre un point fixe lettre \`a lettre on obtient la caract\'erisation suivante qui sera souvent utile.

\begin{lemme}\label{uniforme}
  Si $\sigma$ est un morphisme $q$-uniforme d'un alphabet $\Sigma$ dans lui m\^eme, alors $w$ 
  en est un point fixe si et seulement si :
  \begin{displaymath}
    \forall{n}\in\mathbb{N},\forall{r}\le{q},\sigma(w_{n})_{r}=w_{qn+r}
  \end{displaymath}
\end{lemme}

Le lemme est facile \`a d\'emontrer d'une fa\c con tr\`es intuitive. Les deux suites $t$ et $r$ sont image de point fixe d'un morphisme 2-uniforme. On est \`a pr\'esent en mesure de donner une g\'en\'eralisation qui porte le nom de Cobham :

\begin{proposition}[Cobham, 1972]
  Soient $q\in\mathbb{N^*}$ et $(u_{n})_{n}$ une suite \`a valeurs dans $\{0,\cdots,q-1\}$,
  alors les trois assertions suivantes sont \'equivalentes :
  \begin{enumerate}
  \item la suite $(u_{n})_{n}$ est $q$-automatique
  \item le $q$-noyau $K_u$ de $u$ est fini
  \item la suite $(u_{n})_{n}$ est image d'un point fixe d'un morphisme $q$-uniforme
  \end{enumerate}
\end{proposition}  

\begin{proof}
  $\romannumeral1)\Rightarrow\romannumeral2)$. On dispose d'un $k$-automate $\cal{A}$ = $\{Q,\Sigma,\delta,i,\tau\}$  
  qui engendre $(u_{n})_{n}$. On rappelle que $u_{n}=\tau(\delta(q_{0},\langle{n}\rangle_{q}))$. Or pour $i\ge
  {0},0\le{j}<i$, $(q^{i}n+j)_{q}=j0\ldots0\langle{n}\rangle_{q}=w\langle{n}\rangle_{q}$ o\`u l'on note 
  $w=j0^{i-1}$. On a alors $u_{q^{i}n+j}=\tau(\delta(q_{0},w\langle{n}\rangle_{q}))=\tau(\delta(\delta(q_{0},w),
  \langle{n}\rangle_{q})).$
 
  Comme il y a un nombre fini d'\'etats, ainsi que $\delta(q_{0},w)$ selon que $i$ et
  $j$ varient. Le $q$-noyau est alors bien fini.

  $\romannumeral2)\Rightarrow\romannumeral3)$. Le $q$-noyau \'etant fini suppos\'e de cardinal $d$, on peut alors
  \'ecrire $K_{u}=\{(u_{n}^{(1)})_{n},(u_{n}^{(2)})_{n},\ldots,(u_{n}^{(d)})_{n}\}$ avec $(u_{n}^{(1)})_{n}=(u_{n})
  _{n}$. On d\'efinit une suite $(V_{n})_{n}$ sur l'alphabet $\Sigma^{d}$ par $\forall{n}\in\mathbb{N}$, $V_{n}=
  (u_{n}^{(1)},u_{n}^{(2)},\ldots,u_{n}^{(d)})$. Pour $j\in[0,q-1]$, on dispose d'une application :
  $$\sigma_{j}:\Sigma^{d}\rightarrow\Sigma^{d}\ telle que\ \forall{n}\in\mathbb{N},V_{qn+j}=\sigma_{j}(V_{n})$$
  On consid\`ere le morphisme $\sigma:(\Sigma^{d})^{*}\rightarrow(\Sigma^{d})^{*}$ d\'efini par :
  $$\forall{V}\in\Sigma^{d},\sigma(V)=\sigma_{0}(V)\sigma_{1}(V)\cdots \sigma_{q-1}(V).$$
  
  Il est ais\'e \`a voir que le morphisme $h$ est $q$-uniforme, et par le \textbf{lemme \ref{uniforme}}, la suite
  $(V_{n})_{n} $est point fixe de $\sigma$. $(u_{n})_{n}$ est l'image de $(V_{n})_{n}$ par la projection sur la
  premi\`ere coordonn\'ee. 

  $\romannumeral3)\Rightarrow\romannumeral1)$. On pose $(v_{n})_{n}=\sigma^{\infty}(v)$\footnote{$\sigma^{\infty}
  (v)$ d\'esigne le mot $vx\sigma(x)\sigma^{2}(x)\ldots$ qui est un point fixe de $\sigma$ si $\sigma(v)=vx$ avec
  $x$ non vide.}, c'est-\`a-dire le point fixe de $\sigma$ o\`u $\sigma$ est un morphisme $q$-uniforme sur
  l'alphabet $\Sigma^{'}$ et posons aussi une application $f$ telle que $\forall{n},u_{n}=f(v_{n})$.

  Consi\'erons maintenant l'automate $\cal{A}$ = $\{\Sigma^{'},\Sigma,\delta,v,f\}$ o\`u la fonction de transition
  $\delta$ est d\'efinie par : $\sigma(x)=\delta(x,0)\delta(x,1)\cdots\delta(x,q-1)$. On a alors $\forall{n}\in
  \mathbb{N},u_{n}=f(\delta(v,\langle{n}\rangle_{q}))$. En effet, on montre par r\'ecurrence que :
  $$\forall{k},\sigma^{k}(v)=\delta(v,\langle{0}\rangle_{q})\delta(v,\langle{1}\rangle_{q})\cdots\delta(v,
  \langle{q^{k}-1}\rangle_{q}).$$
  
  L'initialisation est claire. Pour l'h\'eritage, on a :
  \begin{displaymath}
  \begin{array}{ll}
    \sigma^{k+1}(v)&=\sigma(\delta(v,\langle{0}\rangle_{q})\delta(v,\langle{1}\rangle_{q})\cdots\delta(v,
    \langle{q^{k}-1}\rangle_{q}))\\&=\delta(\delta(v,\langle{0}\rangle_{q}),0)\cdots\delta(\delta(v,\langle{0}
    \rangle_{q}),q-1)\cdots\delta(\delta(v,\langle{q^{k}-1}\rangle_{q}),0)\cdots\delta(\delta(v,\langle{q^{k}-1}
    \rangle_{q}),q-1)\\&=\delta(v,\langle{0}\rangle_{q})\cdots\delta(v,\langle{q-1}\rangle_{q})\cdots\delta(v
    ,\langle{q^{k+1}-q}\rangle_{q})\cdots\delta(v,\langle{q^{k+1}-q+(q-1)}\rangle_{q}).
  \end{array}
  \end{displaymath}

  Donc $(u_{n})_{n}$ est bien engendr\'ee par le $q$-automate $\cal{A}$.
\end{proof}

\section{Th\'eor\`eme de Christol}
Nous pr\'esentons dans cette partie un r\'esultat de Gilles Christol. Auparavant, les suites consid\'er\'ees sont $q$-automatiques, avec $q=p^{n}$ et $p$ premier. Le th\'eor\`eme donne une caract\'erisation surprenante du fait d'\^etre automatique en terme d'alg\'ebricit\'e (voir \cite{Allouche05}) de s\'erie sur le corps $\mathbb{F}_{q}(X)$.\footnote{$\mathbb{F}_{q}$ d\'esigne le corps fini \`a $q$ \'el\'ements.} Plus pr\'ecis\'ement, le th\'eor\`eme donne une \'equivalence entre l'alg\'ebricit\'e d'une s\'erie formelle \`a coefficients dans un corps fini et la nature automatique de la suite des coefficients. On va, dans un premier temps, d\'efinir un op\'erateur qui porte le nom de Cartier.

\subsection{Op\'erateur de Cartier}

Notons $\mathbb{F}_{q}[[X]]$ l'ensemble des s\'eries formelles \`a coefficients dans $\mathbb{F}_{q}$. Soit $u$ une suite \`a valeur dans $\mathbb{F}_{q}$, on note : $$F(u)=F_{u}=\sum_{n=0}^{\infty}u_{n}X^{n}.$$

Pour \'eclaircir l'id\'ee, regardons les deux exemples suivantes:

\begin{exemple}
  On se place dans $\mathbb{F}_{2}$\footnote{On confond donc $+$ et $-$, de plus, pour une s\'erie
  formelle $F$, $F(X^{2})=F(X)^{2}$}. En utilisant la d\'efinition de la suite de Thue-Morse $t$, il
  vient : 
  \begin{displaymath}
  \begin{array}{lll}
  F_{t}(X)&=\sum_{n}t_{n}X^{n}&=\sum_{n}t_{2n}X^{2n}+\sum_{n}t_{2n+1}X^{2n+1}\\&&=\sum_{n}t_{n}
  X^{2n}+\sum_{n}X^{2n+1}+\sum_{n}t_{n}X^{2n+1}\\&&=F(X^{2})+\frac{X}{1+X^{2}}+XF(X^{2})\\&&=(1+X)F(X^{2})+
  \frac{X}{1+X^{2}}.
  \end{array}
  \end{displaymath}
  $F$ est donc alg\'ebrique sur le corps $\mathbb{F}_{2}(X)$ des fractions rationnelles modulo 2.
  Elle est racine du polyn\^ome $P$ \`a coefficients dans $\mathbb{F}_{2}(X)$ : $$P(Y)=(1+X)Y^{2}+Y+
  \frac{X}{1+X^{2}}.$$
\end{exemple}

\begin{exemple}
  Par des calculs analogues, on obtient la relation suivante pour la suite de Rudin-Shapiro $r$ : $$(1+X)^
  {5}F(X)^{2}+(1+X)^{4}F(X)+X^{3}=0.$$
\end{exemple}

Avant de continuer, il sera utile d'introduire une famille d'op\'erateurs sur les s\'eries formelles qui portent le nom de l'op\'erateur de Cartier.

\begin{definition}[Cartier]
  Soit $r\in\{0,\cdots,q-1\}.$ On d\'efinit l'op\'erateur $\Lambda_{r}$ : $$\Lambda_{r}:\sum_{n}^{ }a_{n}X^
  {n}\in\mathbb{F}_{q}[[X]]\mapsto\sum_{n}a_{qn+r}X^{n}.$$
\end{definition}

Le lemme suivant se v\'erifie facilement.

\begin{lemme}
  Soit $r\in\{0,\cdots,q-1\}$ et $F,G\in\mathbb{F}_{q}[[X]].$ Alors : $$\Lambda_{r}(FG^{q})=\Lambda_{r}(F)G
  .$$
\end{lemme}

Pr\'ecisons maintenant ce que veut dire \^etre alg\'egrique sur $\mathbb{F}_{q}(X)$ pour une s\'erie formelle $F$.

\begin{lemme}\label{liaison}
  Soit $F=\sum_{n}a_{n}X^{n}$ une s\'erie formelle \`a coefficients dans $\mathbb{F}_{q}$, alors $F$ est
  alg\'ebrique si et seulement si on peut trouver des polyn\^omes $B_{0}(X),B_{1}(X),\cdots,B_{k}(X)$, non
  tous nuls tels que : $$B_{0}F+B_{1}F^{q}+\cdots+B_{k}F^{q^{k}}=0$$ 
  On peut supposer en outre que $B_{0}\neq 0$.
\end{lemme}

\begin{proof}
  Si $F$ est alg\'ebrique sur $\mathbb{F}_{q}(X)$, la famille $(F^{q^{k}})_{k\ge0}$ est li\'ee, d'o\`u une
  relation de combinaison lin\'eaire non triviale de la forme de l'\'enonc\'e. R\'eciproquement une telle 
  liaison implique l'alg\'ebricit\'e de $F$. Il reste \`a montrer que $B_{0}$ n'est pas nul.

  Prenons $k\in\mathbb{N}$ minimal tel que l'on ait une relation de liaison non triviale comme ci-dessus et 
  montrons  par l'absurde que $B_{0}=0$ : $$B_{1}F^{q}+\cdots+B_{k}F^{q^{k}}=0$$
  Puis, en utilisant le lemme plus haut : $$\Lambda_{r}(B_{1}F^{q}+\cdots+B_{k}F^{q^{k}})=\Lambda_{r}(B_{1}
  )F+\cdots+\Lambda_{r}(B_{k})F^{q^{k-1}}=0$$
  Ce qui contredit alors la minimalit\'e de $k$.
\end{proof}

\subsection{Th\'eor\`eme de Christol}

\begin{theoreme}[Christol, 1979]\footnote{La version donn\'ee ici est d\^ue \`a Christol, Kamae, Mend\`es et Rauzy. Dans sa version initiale, le th\'eor\`eme ne portait que sur $\mathbb{F}_{2}$}
  Soit $p\ge2$ un nombre premier et $q$ une puissance de $p$. Une suite $u$ \`a valeurs dans $\mathbb{F}_
  {q}$ est $q$-automatique si et seulement si la s\'erie formelle $F(u)=\sum_{n}u_{n}X^{n}$ est
  alg\'ebrique sur $\mathbb{F}_{q}(X)$.
\end{theoreme}

\begin{proof}
  $(\Rightarrow)$. Soit $u$ une suite $q$-automatique. On se servira de l'\'equivalence prouv\'ee par
  Cobham : le $q$-noyau $K_{u}$ est fini. Posons alors $d=Card(K_{u})$.
  $$F(u)=\sum_{r=0}^{q-1}X^{r}\sum_{n}u_{qn+r}X^{nq}$$
  Comme on a dans $\mathbb{F}_{q}$, $G(X^{q})=G(X)^{q}$ o\`u $G$ est un polyn\^ome
  $$F(u)=\sum_{r=0}^{q-1}X^{r}(\sum_{n}u_{qn+r}X^{n})^{q}$$
  Ce qui montre que :
  $$F(u)\in Vect_{v\in K_{u}}\langle{F(v)^{q}}\rangle$$
  On peut ainsi r\'ep\'eter le processus de la m\^eme mani\`ere :
  $$\forall{k}\le{d}:F(u)^{q^{k}}\in Vect_{v\in K_{u}}\langle{F(v)^{q^{k+1}}}\rangle$$
  Et donc par une r\'ecurrence simple, on obtient :
  $$\forall{k}\le{d}:F(u)^{q^{k}}\in Vect_{v\in K_{u}}\langle{F(v)^{q^{d+1}}}\rangle$$
  Mais $Vect_{v\in K_{u}}\langle{F(v)^{q^{d+1}}}\rangle$ est de dimension au plus $d$, la famille $\{F(u),F(u)^{q},
  \cdots,F(u)^{q^{d}}\}$ est li\'ee. $F(u)=\sum_{n}u_{n}X^{n}$ est donc alg\'ebrique sur $\mathbb{F}_{q}(X)$.

  $(\Leftarrow)$. R\'eciproquement, si la s\'erie formelle $F(u)$ est alg\'ebrique sur $\mathbb{F}_{q}(X)$, d'apr\`es
  le $\mathbf{lemme\ \ref{liaison}}$, on peut trouver des poly\^omes $B_{i}$ avec $B_{0}$ non nul tels que :
  $$\sum_{i=0}^{k}B_{i}(X)F_{u}(X)^{q^{i}}=0$$
  Posons $G_{u}=F_{u}/B_{0}$. Il vient $G_{u}=\sum_{i=0}^{k}C_{i}G_{u}^{q^{i}}$ o\`u $\forall{i}\in\{1,\ldots,k\},C_{i}
  =-B_{i}B_{0}^{q^{i}-2}$. 
  Posons ensuite $N=max\{deg(B_{0}),deg(C_{1}),\cdots,deg(C_{k})\}$ et $\cal{H}$ l'ensemble d\'efini par :
  \begin{displaymath}
  \begin{array}{cc}
    \cal{H}&=\{H\in\mathbb{F}_{q}[[X]]|H=\sum_{i=0}^{k}D_{i}G_{u}^{q^{i}} et D_{i}\in\mathbb{F}_{q}
    [X],deg(D_{i})\le{N}\}.
  \end{array}
  \end{displaymath}
  $\cal{H}$ est un ensemble fini contenant $F_{u}$. Pour montrer que $K_{u}$ est fini, il suffit de montrer
  que $\forall{r}<q$, $\cal{H}$ est stable par $\Lambda_{r}$. Soit alors $H$ un \'el\'ement de $\cal{H}$, on a :
  $$\Lambda_{r}(H)=\Lambda_{r}(D_{0}G_{u}+\sum_{i=1}^{k}D_{i}G_{u}^{q^{i}})=\Lambda_{r}(\sum_{i=1}^{k}
  (D_{0}G_{u}+D_{i})G_{u}^{q^{i}})=\sum_{i=1}^{k}\Lambda(D_{0}C_{i}+D_{i})G_{u}^{q^{i-1}}$$
  Comme pour tout $i$, on a $deg(D_{0}C_{i}+D_{i})\le{deg(D_{0}C_{i}+D_{i})}/{q}\le{2N}/{q}\le{N}$, donc
  $\Lambda_{r}(H)\in$ $\cal{H}$.
\end{proof}

\section{Divers}
Dans cette partie, nous donnons quelques autres r\'esultats importants sur les suites automatiques. Cobham a d\'emontr\'e le r\'esultat qui suit dont on trouvera une d\'emonstration dans \cite{Allouche03} :

\begin{theoreme}
  Soit $k$ et $l$ deux entiers multiplicativement ind\'ependants\footnote{$i.e\ k^{\alpha}=l^{\beta}
  \Rightarrow{\alpha=\beta=0}$} et soit $u$ une suite $k$ et $l$-automatique, alors $u$ est
  ultimement p\'eriodique.
\end{theoreme}

\begin{corollaire}
  Soit $q_{1}$ et $q_{2}$ multiplicativement ind\'ependants et soit $u$ une suite telle que $F_{u}$ 
  soit \`a la fois alg\'ebrique sur $\mathbb{F}_{q1}$ et $\mathbb{F}_{q2}$, alors $u$ est ultimement
  p\'eriodique.
\end{corollaire}

\begin{proof}
  En vertu du th\'eor\`eme pr\'ec\'edent et celui de Christol.
\end{proof}

En particulier, les suites de Thue-Morse et de Rudin-Shapiro ne sont pas 3-automatiques.\footnote{Par contrapos\'ee.} Il est donc int\'eressant de comparer ce corollaire \`a la conjecture suivante.

\begin{conjecture}
  Soit $(u_{n})_{n}\in\{0,1\}^{\mathbb{N}}$ telle que les deux nombres r\'eels $\sum_{n\in{N}}\frac{u_{n}}{2^{n}}$ et $
  \sum_{n\in{N}}\frac{u_{n}}{3^{n}}$ sont alg\'ebriques sur $\mathbb{Q}$, alors ces deux nombres sont rationnels.
\end{conjecture}

Par ailleurs, on montre que si deux suites $u$ et $v$ \`a valeurs dans $\{0,\ldots,q-1\}$ sont $q$-automatiques, leur produit $uv$ l'est aussi. En appliquant le th\'eor\`eme de Christol, on a :

\begin{corollaire}
  Soient $u$ et $v$ \`a valeurs dans $\{0,\ldots,q-1\}$ telles que $F_{u}$ et $F_{v}$ sont alg\'ebriques sur $\mathbb
  {F}_{q}(X)$, alors le produit d'Hadamard\footnote{$i.e$ produit composante par composante.} de $F_{u}$ et $F_{v}$ est 
  aussi alg\'ebrique que l'on note $F_{uv}$.
\end{corollaire}

Le lien entre les automates et l'alg\'ebricit\'e laisse entrevoir un champ de recherche en th\'eorie des nombres et en alg\`ebre.

% Bibliographie
% Style de la bibliographie
% Nom du fichier de bibiographie (sans le .bib)
\bibliographystyle{plain}
\bibliography{suites_automatiques}

\end{document}
